%%=============================================================================
%% Methodologie
%%=============================================================================

\chapter{Methodologie}
\label{ch:methodologie}

In Rapporteringsnood (hoofdstuk \ref{ch:rapporteringsnood}) wordt de te implementeren Key Performance Indicator (KPI) toegelicht. Er wordt in detail overlopen wat de KPI betekent en wat er exact moet berekend worden.

Daara zal het onderzoek opgedeeld worden in twee onderdelen: het voorbereidende werk en het vergelijkend onderzoek. 

In het voorbereidende werk beschrijven we wat de benodigdheden zijn voor dit experiment en worden twee data warehouses stap voor stap opgebouwd. De ene data warehouse gebaseerd op het dimensioneel model, de andere data warehouse gebaseerd op Data Vault. Het gehele ETL-proces (\ref{sec:etl}) wordt neergepend in deze paper voor zowel Data Vault als voor het dimensioneel model. Hierdoor kan het experiment (indien gewenst) nagebootst worden. De gebruikte data-files zal als bijlage toegevoegd worden aan dit document.

In het vergelijkend onderzoek zal de onderzoeksvraag en deelvragen beantwoord worden. 