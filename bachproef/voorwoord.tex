%%=============================================================================
%% Voorwoord
%%=============================================================================

\chapter*{Woord vooraf}
\label{ch:voorwoord}

%% TODO:
%% Het voorwoord is het enige deel van de bachelorproef waar je vanuit je
%% eigen standpunt (``ik-vorm'') mag schrijven. Je kan hier bv. motiveren
%% waarom jij het onderwerp wil bespreken.
%% Vergeet ook niet te bedanken wie je geholpen/gesteund/... heeft

Voor het bekomen van een professionele bachelor in de Toegepaste Informatica dient een bachelorproef geschreven te worden. 

Het onderwerp van deze bachelorproef werd mij aangebracht door Jochen Stroobants bij de start van mijn stage. Als stageopdracht diende een data warehouse opgesteld te worden aan de hand van Data Vault 2.0. Zelf was ik kritisch tegenover deze keuze en ik vroeg mij dan ook af of Data Vault 2.0 de juiste keuze was ten opzichte van het dimensioneel model. Het leek mij zeer interessant om de vergelijking te maken tussen beide modellen en om te concluderen of het dimensioneel geen betere keuze was.

Graag bedank ik in eerste instantie mijn promotor Stijn Lievens voor de uitgebreide feedback die ik ontving tijdens het opstellen van een voorstel en tijdens het schrijven van deze paper. Door zijn kritische kijk kon ik de kwaliteit van deze paper naar een hoger niveau brengen. 

Anderzijds bedank ik graag het bedrijf Cubis Solutions, omdat ik gebruik kon en mocht maken van hun infrastructuur. In het bijzondere wil ik graag Jochen Stroobants, Sven Van Rillaer en Sander Allert bedanken voor hun kennis en feedback die ik ontving tijdens het opstellen van deze paper.