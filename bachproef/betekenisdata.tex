%%=============================================================================
%% Onderzoek
%%=============================================================================
\chapter{Betekenis van de brondata}
\label{ch:betekenisdata}
In dit hoofdstuk wordt beschreven wat de betekenis is van de data die gebruikt wordt in dit onderzoek. Zo kan de data goed geïnterpreteerd worden. De data die gebruikt wordt is gegenereerde data, en is niet afkomstig uit een bronsysteem van DHL Pharma Logistics.

\section{Overzicht betekenissen}
\begin{center}
	\renewcommand{\arraystretch}{2}%
	\begin{longtable}{  l  l  p{6cm} }
		\textbf{Kolomnaam} & \textbf{Afkomst} & \textbf{Betekenis} \\ \hline
		Data Vault\_ENTITY\_ID & Customer\_entities.csv & Een uniek identificatie nummer voor een Data Vault entiteit. 
		Een Data Vault entiteit is een vestiging van een bepaald bedrijf.  \\ \hline
		Data Vault\_ENTITY\_NAME & Customer\_entities.csv & De vestigingsnaam van de entiteit.  \\ \hline
		Data Vault\_GROUP\_ID & Customer\_entities.csv & Identificatienummer van de overkoepelende groep.  \\ \hline
		Data Vault\_GROUP\_ID & Customer\_groups.csv & Identificatienummer van de groep.
		Een groep bestaat uit meerdere entiteiten.  \\ \hline
		Data Vault\_GROUP\_NAAM & Customer\_groups.csv & De naam voor de groep.  \\ \hline
		STAFF\_ID & Staff.csv & Personeelsnummer van een werknemer.  \\ \hline
		STAFF\_NAME & Staff.csv & De naam van een werknemer.  \\ \hline
		WAREHOUSE\_ID & Staff.csv & Het warenhuis waar de werknemer actief is.  \\ \hline
		WAREHOUSE\_ID & Warehouses.csv & Het identificatienummer voor een bepaalde warenhuis.  \\ \hline
		WAREHOUSE\_NAME & Warehouses.csv & De gemeente/stad van het warenhuis (tevens ook de naam).  \\ \hline
		PRODUCT\_ID & Products.csv & Het artikelnummer. Dit is een uniek nummer.  \\ \hline
		PRODUCT\_DESCRIPTION & Products.csv & Naam/beschrijving van een bepaald product.  \\ \hline
		Data Vault\_ENTITY\_ID & Products.csv & Het identificatienummer van de Data Vault entiteit waarvan het product afkomstig is.  \\ \hline
		STATUS\_ID & Dock\_to\_stock\_status.csv & Identificatienummer van een bepaalde status. 
		Een status wordt gegeven aan een pallet wanneer deze gestockeerd wordt. Ofwel is deze op tijd gestockeerd,
		ofwel is er een bepaalde reden waarom dit niet op tijd kon gestockeerd worden (bijvoorbeeld een te druk moment).  \\ \hline
		STATUS\_DESCRIPTION & Dock\_to\_stock\_status.csv & Beschrijving/uitleg over de status.  \\ \hline
		PRODUCT\_ID & Products.csv & Het artikelnummer. Dit is een uniek nummer.  \\ \hline
		LEVERAGE\_ID & Leverages.csv & Een nummer die toegekend wordt aan een bepaalde levering.
		Vaak ook een referentienummer die de klant (Data Vault) ziet op de factuur. \\ \hline
		PALLET\_ID & Leverages.csv & Het identificatienummer van een pallet.
		Een levering kan uit meerdere palletten bestaan. \\ \hline
		STAFF\_ID & Leverages.csv & Een personeelsnummer. 
		Dit personeelslid was verantwoordelijk voor het juist en tijdig stockeren van een pallet. \\ \hline
		PRODUCT\_ID & Leverages.csv & Het identificatienummer van het product dat gestockeerd werd.
		Er wordt enkel 1 productsoort op een pallet gestockeerd. \\ \hline
		PRODUCT\_QUANTITY & Leverages.csv & De hoeveelheid producten aanwezig op een pallet.
		De hoeveelheid wordt weergegeven per unit. \\ \hline
		DOCK\_TIME & Leverages.csv & Het tijdstip waarop de pallet toekomt in het magazijn. \\ \hline
		STOCK\_TIME & Leverages.csv & Het tijdstip waarop de pallet gestockeerd werd op de juiste plaats. \\ \hline
		STATUS\_ID & Leverages.csv & Identificatienummer voor de status. Indien deze tijdig werd gestockeerd, dan werd geen status toegekend. \\
	\caption{Betekenis van de gebruikte data in dit experiment.}
	\label{tab:betekenisdata}
	\end{longtable}
\end{center}