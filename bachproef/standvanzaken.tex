\chapter{Stand van zaken}
\label{ch:stand-van-zaken}

% Tip: Begin elk hoofdstuk met een paragraaf inleiding die beschrijft hoe
% dit hoofdstuk past binnen het geheel van de bachelorproef. Geef in het
% bijzonder aan wat de link is met het vorige en volgende hoofdstuk.

% Pas na deze inleidende paragraaf komt de eerste sectiehoofding.

Dit hoofdstuk bevat een literatuurstudie omtrent data warehousing. Na het lezen van dit hoofdstuk zullen begrippen zoals dimensioneel modelleren, data vault 2.0 en data warehousing jou niet meer onbekend zijn. Ook zullen beide modelleertechnieken dieper bekeken worden.

%\textcite{Knuth1998}
\section{Inleiding data warehouse}

\epigraph{Torture the data, and it will confess to anything. }{\textit{Ronald Coase \\ Winnaar Nobelprijs in Economie (1991)}}

Veel moderne, digitale bedrijven genereren tegenwoordig enorme volumes data. Deze data kan afkomstig zijn uit verschillende bronnen: CRM-systeem, flat-files (vb. rekenbladen), Twitter-feeds, ... Bestuursleden gebruiken deze data om beslissingen te nemen die de onderneming toelaat om te (blijven) groeien of om bepaalde problemen op te sporen. Stel dat een onderneming meer kosten maakt dan opbrengsten. Op basis van alle gegevens die het bedrijf bezit, kan hieruit dan een analyse gemaakt worden. Zijn er overbodige kosten? Worden onze producten/diensten aan een te lage prijs verkocht? Dit zijn maar enkele vragen die kunnen opgelost worden wanneer het bestuur de correcte rapporteringen ontvangt. 

\subsection{Wat is een data warehouse?}
De definitie van een data warehouse luidt als volgt: \textit{"een subject-georiënteerde, geïntegreerde, tijd-variante, niet-vluchtige collectie van gegevens dat in eerste instantie gebruikt wordt bij organisaties om beslissingen te nemen"} ~\autocite{Panos2000}.

\paragraph{Subject-georiënteerd}
Dit begrip slaat op het feit dat een data warehouse met de reden gebouwd is om data te analyseren, niet om transacties op toe te passen. Dit wordt uitgebreid besproken in subsectie \ref{sec:oltp-vs-olap}.


\paragraph{Geïntegreerd}
Dit betekent dat de data warehouse een "centrale" databank is die gegevens bevat vanuit verschillende bronsystemen (bijvoorbeeld gegevens uit het klantenbestand en gegevens uit het verkoopsysteem). Deze data kan effectief ingeladen worden, maar ook opgeslagen worden in virtuele tabellen.

\paragraph{Tijd-variant}
Alle data van het verleden, moet terug te vinden zijn in de data warehouse. Dit betekent dat data uit het verleden (bijvoorbeeld een vorig adres van een klant) moet beschikbaar zijn, ook al is deze in het transactioneel systeem aangepast.

\paragraph{Niet-vluchtig}
De data die in het systeem zit, moet onveranderlijk zijn, ook al zijn deze fout. Om de foutieve data toch aan te passen, zal er een nieuwe rij moeten toegevoegd worden die de juiste data bevat, die een hogere versie bevat dan de vorige rij. 

\paragraph{Conclusie}
We kunnen dus uit de definitie van een data warehouse afleiden dat het een grote databron is die alle gegevens bevat die een organisatie bezit vanaf het moment waarbij de data warehouse geïmplementeerd is tot het heden. Op deze databron worden dan analyses gemaakt.

\subsection{Waarom is er nood aan een data warehouse?}
Volume data
apart van transactioneel sys

\subsection{Wat is het doel van een data warehouse?}
Lalalalalalallalala

\subsection{Wat is OLTP en wat zijn de verschillen met OLAP?}
\label{sec:oltp-vs-olap}
Lalalalalalallalala

\subsection{Wat zijn de benodigdheden voor een data warehouse?}
Lalalalalalallalala

\subsubsection{Methodologie}
Lalalalalalallalala

\subsubsection{Architectuur}
Lalalalalalallalala

\subsubsection{Technologie}
Lalalalalalallalala

\paragraph{Databank}
Lalalalalalallalala

\paragraph{Data integratie software}
Lalalalalalallalala

\section{Dimensioneel modelleren via Kimball}

\subsection{Inleiding}
Lalalalalalallalala

\subsection{Architectuur}
Lalalalalalallalala

\subsubsection{Staging area}
Lalalalalalallalala

\subsubsection{Data warehouselaag}
Lalalalalalallalala

\subsection{Componenten}
Lalalalalalallalala

\subsubsection{Dimenties}
Lalalalalalallalala

\subsubsection{Facts}
Lalalalalalallalala

\subsubsection{Sterschema's}
Lalalalalalallalala

\section{Modelleren via Data Vault 2.0}

\subsection{Inleiding}
Lalalalalalallalala

\subsection{Architectuur}
Lalalalalalallalala

\subsubsection{Staging area}
Lalalalalalallalala

\subsubsection{Raw data vault}
Lalalalalalallalala

\subsubsection{Business vault}
Lalalalalalallalala

\subsubsection{Information marts}
Lalalalalalallalala

\subsection{Componenten}
Lalalalalalallalala

\subsubsection{Hubs}
Lalalalalalallalala

\subsubsection{Links}
Lalalalalalallalala

\subsubsection{Sattelieten}
Lalalalalalallalala

\section{Rapporteringsomgevingen}
Inleiding enzoooooo

\subsection{SAP Analytics Cloud}
Lalalalalalallalala

\subsection{Power BI}
Lalalalalalallalala

\subsection{Andere omgevingen}
Lalalalalalallalala





