%%=============================================================================
%% Conclusie
%%=============================================================================

\chapter{Conclusie}
\label{ch:conclusie}

%% TODO: Trek een duidelijke conclusie, in de vorm van een antwoord op de
%% onderzoeksvra(a)g(en). Wat was jouw bijdrage aan het onderzoeksdomein en
%% hoe biedt dit meerwaarde aan het vakgebied/doelgroep? Reflecteer kritisch
%% over het resultaat. Had je deze uitkomst verwacht? Zijn er zaken die nog
%% niet duidelijk zijn? Heeft het onderzoek geleid tot nieuwe vragen die
%% uitnodigen tot verder onderzoek?

Voor DHL Pharma Logistics werd gekozen om een data warehouse te ontwerpen aan de hand van de Data Vault methodologie. Was dit de juiste keuze? Was het dimensioneel model in hun project geen betere keuze? Wat wijst deze vergelijkende studie uit?

Het onderzoek wijst uit dat het dimensioneel model een betere keuze was voor het modelleren van de data voor DHL Pharma Logistics. Dit is bovendien ook het resultaat dat ik verwacht had.

Het dimensioneel model voert sneller leesresultaten uit in vergelijking met het Data Vault, dit omdat het Data Vault model meer relaties heeft. Hierdoor zullen veel meer joins moeten gebeuren wanneer alle data moet opgehaald worden. 

Bij de Data Vault methodologie wordt er meer informatie (Hash keys, informatie over extractie, ..) en tabellen opgeslagen in een databank. Dit zorgt ervoor dat de volumes van data enorm stijgen. Bijgevolg zal er dus een hogere kostprijs zijn om deze data te stockeren. Aangezien dit geen requirement is voor DHL Pharma Logistics, zou dit leiden naar een onnodige meerkost voor dit project.

De KPI's waarvoor een model moet opgesteld worden zijn gestandaardiseerd, en dienen niet flexibel te zijn. Indien de berekening voor de KPI's zouden gewijzigd worden, hoeven enkel sommige parameters uit het ETL-proces aangepast te worden en niet het datamodel zelf.

Het grote nadeel voor het implementeren van een project aan de hand van het dimensioneel model, is dat er niet kan gewerkt worden via de agile manier.

Er kan verder onderzoek verricht worden naar hoe het Data Vault model omspringt met NoSQL-data en big data in het algemeen. Is het een betere keuze om een Data Vault model te gebruiken bij big data modellen (aangezien Data Vault parallel inladen van data toelaat)? 