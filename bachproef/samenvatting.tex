%%=============================================================================
%% Samenvatting
%%=============================================================================

% TODO: De "abstract" of samenvatting is een kernachtige (~ 1 blz. voor een
% thesis) synthese van het document.
%
% Deze aspecten moeten zeker aan bod komen:
% - Context: waarom is dit werk belangrijk?
% - Nood: waarom moest dit onderzocht worden?
% - Taak: wat heb je precies gedaan?
% - Object: wat staat in dit document geschreven?
% - Resultaat: wat was het resultaat?
% - Conclusie: wat is/zijn de belangrijkste conclusie(s)?
% - Perspectief: blijven er nog vragen open die in de toekomst nog kunnen
%    onderzocht worden? Wat is een mogelijk vervolg voor jouw onderzoek?
%
% LET OP! Een samenvatting is GEEN voorwoord!

%%---------- Nederlandse samenvatting -----------------------------------------
%
% TODO: Als je je bachelorproef in het Engels schrijft, moet je eerst een
% Nederlandse samenvatting invoegen. Haal daarvoor onderstaande code uit
% commentaar.
% Wie zijn bachelorproef in het Nederlands schrijft, kan dit negeren, de inhoud
% wordt niet in het document ingevoegd.

\IfLanguageName{english}{%
\selectlanguage{dutch}
\chapter*{Samenvatting}
\selectlanguage{english}
}{}

%%---------- Samenvatting -----------------------------------------------------
% De samenvatting in de hoofdtaal van het document

\chapter*{\IfLanguageName{dutch}{Samenvatting}{Abstract}}

DHL Pharma Logistics wil een data warehouse aanmaken om hun rapporteringen te automatiseren. De data warehouse wordt gemodelleerd via de Data Vault 2.0 methodologie. In deze paper wordt onderzocht of modelleren via deze manier wel degelijk de juiste keuze was ten opzichte van het dimensioneel model.

Deze paper kan nuttig zijn voor informatici die actief zijn of interesse hebben in Business Intelligence. Er is geen specifieke voorkennis rond Data Vault of het dimensioneel modelleren nodig om door deze paper te gaan. Dit vergelijkend onderzoek kan informatici helpen bij het beslissen van een data model methodologie.

Vooraleer het onderzoek van start gaat, wordt een literatuurstudie weergegeven met de informatie en begrippen die nodig zijn om het volledige onderzoek mee te kunnen volgen.

Vervolgens wordt in het onderzoek twee data warehouses opgebouwd op basis van dezelfde data. Er zal een data warehouse gemodelleerd worden via Data Vault 2.0, de andere data warehouse zal gebaseerd zijn op het dimensioneel model. Deze data warehouses zullen van later in dit onderzoek gebruikt worden om beide data modellen te vergelijken op basis van performantie en schaalbaarheid.

Het vergelijkend onderzoek wordt uitgevoerd op basis van vijf pijlers: performantie, complexiteit, flexibiliteit, schaalbaarheid en audit. Zowel Data Vault 2.0 als het dimensioneel model worden onderworpen aan deze vijf pijlers, op basis van de noden van DHL Pharma Logistics wordt een juiste conclusie opgemaakt. 

Als resultaat in dit onderzoek blijkt dat het kiezen voor het opstellen van een data warehouse aan de hand van dimensioneel modelleren een betere keuze was geweest.

Naar de toekomst toe kan er onderzocht worden of Data Vault 2.0 een juiste keuze kan zijn bij het opstellen van big data modellen.